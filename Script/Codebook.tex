\documentclass[12pt]{article}
%\usepackage[latin1]{inputenc}
\usepackage[utf8]{inputenc}
\usepackage[brazil]{babel}
\usepackage[{left=3cm,right=2cm,top=3cm,bottom=2cm}]{geometry}
\usepackage{graphicx}
\usepackage[hidelinks]{hyperref}
\usepackage{float} %Me possibilita usar o Hear para a figura.
\usepackage[dvipsnames]{xcolor}
\pagestyle{empty}

\title{\textbf{CODEBOOK - Lei é Lei?}}


\date{ }

\begin{document}
\maketitle

\noindent SILVA JÚNIOR, J. A.; NASCIMENTO, Willber S.; LIMA, A. F., OMENA, W. S. (2020)  
Lei é lei? Maurice Duverger e as eleições para o Senado no Brasil.Revista Debates. 14, 2, p.153-180.

\tableofcontents
\pagebreak	





	\section{Informações aos Usuários}
	Bancos de dados: Lei é lei? Maurice Duverger e as eleições para o Senado no Brasil.
	A base deste artigo esta disponibilizada em arquivos em .csv no site {\color{red}\url{https://osf.io/sg6dm/}}. Os códigos e rótulos estão disponíveis neste documento. O artigo encontra-se publicado pela revista \textbf{Debates}.
	
	
	\section{Informações Técnicas}
	\begin{tabular}{p{3cm} p{10cm}} 
		Banco 1     & Lei \\
		N           &  162 \\ 
		Variáveis   &  10 \\
		Extensão    & .csv \\  
		Fonte       & TSE  \\  
	\end{tabular} \\
   -- {\color{blue}\url{http://www.tse.jus.br/}}
 \\ \\ \\
	\begin{tabular}{p{3cm} p{10cm}} 
		Banco 2      & part \\
		N           &  86 \\ 
		Variáveis   &  6 \\
		Extensão    & .csv \\  
		Fonte       & TSE 
	\end{tabular} \\ 
-- {\color{blue}\url{http://www.tse.jus.br/}} \\  
\\ \\
	\begin{tabular}{p{3cm} p{10cm}} 
		Banco 3     & Lan \\
		N           &  3 \\ 
		Variáveis   &  3 \\
		Extensão    & .csv \\  
		Fontes      &  TSE 	\end{tabular} \\ 
	-- {\color{blue}\url{http://www.tse.jus.br/}} \\  


	\section{Banco 1 - Lei}
	
		\subsection{Ano}
		{\bf Tipo:} Discreta \\
		{\bf Descrição} Ano da Eleição \\
		\subsection{uf}
		{\bf Tipo:} Nominal \\
		{\bf Descrição:} Sigla do Estado. 
		\subsection{Nep}
		{\bf Tipo:} Contínua \\
		{\bf Descrição:} Razão entre 1.00 o somatório do quadrado das proporções de votos.\\
	
		
		\subsection{Magnitude}
		{\bf Tipo:} Nominal \\
		{\bf Descrição:} Número de vagas em disputa a cada eleição.
		.
		\subsection{Ncand}
		{\bf Tipo:} Numérica \\
		{\bf Descrição:} Número de Candidatos para o Senado.\\
		
		\subsection{RazPerd}
		{\bf Tipo:} Contínua \\
		{\bf Descrição:} Razão entre os percentuais de votos do 1º e do 2º perdedor.\\
		
	
		
		\subsection{PercVal}
		{\bf Tipo:} Contínua \\
		{\bf Descrição:} Percentual de votos válidos para o Senado.\\
		
		\subsection{ConcPres}
		{\bf Tipo:} Contínua \\
		{\bf Descrição:} Somatório do quadrado das proporções de votos para presidente.\\
		
		\subsection{ConcGov}
		{\bf Tipo:} Contínua \\
		{\bf Descrição:} Somatório do quadrado das proporções de votos para Governador.\\
		
		\section{Banco 2 - part}
	
		\subsection{Magnitude}
		{\bf Tipo:} Nominal \\
		{\bf Descrição:} Tamanho da Renovação do Senado (Um Terço e Dois Terços).\\
		
		\subsection{Ano}
		{\bf Tipo:} Discreta \\
		{\bf Descrição} Ano da Eleição \\
		
		\subsection{Partido}
		{\bf Tipo:} Nominal \\
		{\bf Descrição} Nome do partido que aparece no grupo dos principais competidores.\\
		
		\subsection{Número}
		{\bf Tipo:} Discreta \\
		{\bf Descrição}Total de vezes que o partido aparece no grupo de principais competidores.\\
		
		\subsection{Perc}
		{\bf Tipo:} Contínua \\
		{\bf Descrição} Percentual de vezes que o partido aparece no grupo de principais competidores.\\
		
		
		
		
		\section{Banco 3 - Lan}
		\subsection{Ano}
		{\bf Tipo:} Discreta \\
		{\bf Descrição} Ano da Eleição \\
		
		\subsection{LanUF}
		{\bf Tipo:} Contínua \\
		{\bf Descrição:} Percentual de UF onde um partido lançou dois candidatos (eleições dois terços).
		
		\subsection{LanPart}
		{\bf Tipo:} Contínua \\
		{\bf Descrição:} Percentual de Partidos onde um partido lançou dois candidatos (eleições dois terços).\\
		
	
		
	\section{Como citar esse banco de dados}
	Para citar os dados e o artigo utilize sua versão de publicação na página da revista  \textbf{Debates}
	Para citar este codebook, indique a seguinte referência.
	\\ \\
	SILVA JÚNIOR, J. A.; NASCIMENTO, Willber S.; LIMA, A. F., OMENA, W. S. (2020)  
	Lei é lei? Maurice Duverger e as eleições para o Senado no Brasil. Grupo de Pesquisa Cidadania e Políticas Públicas (GCPP/UFAL). Disponível em:	

	
	 \noindent  https://osf.io/sg6dm/  \\
	 https://dataverse.org/ \\ Acesso em: dia/mês/ano. \\
	 
	\section{Pesquisadores - Elaboração e Contato}
		José Alexandre - UFAL \\
		{\color{blue}\href{mailto}{jasjunior2007@yahoo.com.br}} 
		\\ 
		
		\\
		
		 \noindent Willber Nascimento - UFPE \\
		{\color{blue}\href{mailto}{dalsonbritto@yahoo.com.br}}
		\\ \\
		Albany Ferreira - UFMG \\
		{\color{blue}\href{mailto}{albanyinformatica@gmail.com}}
		\\ \\
		Widyane Omena - UFMG \\
		{\color{blue}\href{mailto}{widyaneso@hotmail.com}}
		\\ \\
	

\end{document}